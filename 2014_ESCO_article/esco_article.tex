%% 
%% Copyright 2007, 2008, 2009 Elsevier Ltd
%% 
%% This file is part of the 'Elsarticle Bundle'.
%% ---------------------------------------------
%% 
%% It may be distributed under the conditions of the LaTeX Project Public
%% License, either version 1.2 of this license or (at your option) any
%% later version.  The latest version of this license is in
%%    http://www.latex-project.org/lppl.txt
%% and version 1.2 or later is part of all distributions of LaTeX
%% version 1999/12/01 or later.
%% 
%% The list of all files belonging to the 'Elsarticle Bundle' is
%% given in the file `manifest.txt'.
%% 
%% Template article for Elsevier's document class `elsarticle'
%% with harvard style bibliographic references
%% SP 2008/03/01

%\documentclass[preprint,12pt]{elsarticle}
 \documentclass{elsarticle}
%\documentclass[3p,12pt,authoryear]{elsarticle}

%% Use the option review to obtain double line spacing
%% \documentclass[authoryear,preprint,review,12pt]{elsarticle}

%% Use the options 1p,twocolumn; 3p; 3p,twocolumn; 5p; or 5p,twocolumn
%% for a journal layout:
%% \documentclass[final,1p,times,authoryear]{elsarticle}
%% \documentclass[final,1p,times,twocolumn,authoryear]{elsarticle}
%% \documentclass[final,3p,times,authoryear]{elsarticle}
%% \documentclass[final,3p,times,twocolumn,authoryear]{elsarticle}
%% \documentclass[final,5p,times,authoryear]{elsarticle}
%% \documentclass[final,5p,times,twocolumn,authoryear]{elsarticle}

\usepackage{hyperref}
\hypersetup{
  colorlinks   = true, %Colours links instead of ugly boxes
  urlcolor     = blue, %Colour for external hyperlinks
  linkcolor    = blue, %Colour of internal links
  citecolor   = red %Colour of citations
}

%% For including figures, graphicx.sty has been loaded in
%% elsarticle.cls. If you prefer to use the old commands
%% please give \usepackage{epsfig}
\usepackage{subfig}

%tables
\usepackage{booktabs}

%% The amssymb package provides various useful mathematical symbols
\usepackage{amssymb}
\usepackage{amsmath}
\usepackage{esint}

%% The amsthm package provides extended theorem environments
%% \usepackage{amsthm}

%% The lineno packages adds line numbers. Start line numbering with
%% \begin{linenumbers}, end it with \end{linenumbers}. Or switch it on
%% for the whole article with \linenumbers.
%% \usepackage{lineno}

% just for our notes
\usepackage[usenames,dvipsnames]{color}   %colors


\journal{Applied Mathematics and Computation}

%commands:
%\newcommand{\defref}[1]{\hyperref[#1]{Def.~\ref{#1}}}
\newcommand{\prob}[1]{Problem~{#1}}
\newcommand{\fig}[1]{\hyperref[#1]{Figure \ref{#1}}}
\newcommand{\figpath}{../graphics/}

%math:
\def\vc#1{\mathbf{\boldsymbol{#1}}}     % vector
\def\abs#1{\left|#1\right|}
\def\avg#1{\langle#1\rangle}
\def\d{\mathrm{d}}
\def\norm#1{\| #1 \|}
\def\abs#1{| #1 |}
\def\prtl{\partial}
\newcommand{\dd}{\; \mathrm{d}}
\newcommand{\R}{\mathbf{R}}
\newcommand{\bx}{\vc{x}}
\newcommand*\rfrac[2]{{}^{#1}\!/_{#2}}


\newcommand{\noteJB}[1]{{\color{Blue} \textbf{JB: } \textit{#1}}}
\newcommand{\notePE}[1]{{\color{Orange} \textbf{PE: } \textit{#1}}}

\newdefinition{mdef}{Definition}%[section]

\begin{document}

\begin{frontmatter}

%% Title, authors and addresses

%% use the tnoteref command within \title for footnotes;
%% use the tnotetext command for theassociated footnote;
%% use the fnref command within \author or \address for footnotes;
%% use the fntext command for theassociated footnote;
%% use the corref command within \author for corresponding author footnotes;
%% use the cortext command for theassociated footnote;
%% use the ead command for the email address,
%% and the form \ead[url] for the home page:
%% \title{Title\tnoteref{label1}}
%% \tnotetext[label1]{}
%% \author{Name\corref{cor1}\fnref{label2}}
%% \ead{email address}
%% \ead[url]{home page}
%% \fntext[label2]{}
%% \cortext[cor1]{}
%% \address{Address\fnref{label3}}
%% \fntext[label3]{}

\title{Partition of unity methods for approximation of point water sources in~porous media}

%% use optional labels to link authors explicitly to addresses:
%% \author[label1,label2]{}
%% \address[label1]{}
%% \address[label2]{}

\author[adr]{Pavel Exner\corref{cor1}}
\ead{pavel.exner@tul.cz}
%\ead[url]{https://github.com/Paulie14/xfem\_project}
\cortext[cor1]{Corresponding author.}

\author[adr]{Jan B{\v r}ezina}
\ead{jan.brezina@tul.cz}

\address[adr]{Technical University of Liberec, Studentsk{\' a} 1402/2, 461 17 Liberec 1, Czech Republic}


\begin{abstract}
%% Text of abstract
% In this work we demonstrate the usage of Partition of Unity (PU) methods to improve approximation of singularities 
% in the solution of the Poisson equation. Our model describes a steady flow of water in a system of aquifers
% which consist of porous media. The aquifers are perforated by wells and boreholes which are often represented
% as point sources considering their small diameter in comparison with the vast size of the aquifer. This 
% brings singularities into the solution. The extended and stable generalized finite element method 
% (XFEM and SGFEM) were implemented to solve the problem and a proper adaptive integration strategy was 
% developed to gain optimal convergence rates.

Several partition of unity methods (PUM) are compared on the problem of steady water flow in an aquifer-well system. 
In order to improve the approximation of a singular behavior of the pressure near the wells, the standard finite element space is enriched with
the cut-off fundamental solution to the Laplace problem. The optimal order of 
convergence of PUM in terms of $L^2$ norm of the error is demonstrated.
The error of adaptive integration is analyzed and a~new adaptive strategy is proposed. The influence of the choice
of the enriched domain is investigated and its impact on the error is demonstrated numerically.

\end{abstract}

\begin{keyword}
%% keywords here, in the form: keyword \sep keyword
Extended finite element method \sep 
Groundwater flow \sep
Adaptive integration \sep 
Singular solution 

%% PACS codes here, in the form: \PACS code \sep code
\PACS 02.60.Lj \sep        %Ordinary and partial differential equations; boundary value problems
\PACS 02.60.Jh             %Numerical differentiation and integration

%% MSC codes here, in the form: \MSC code \sep code
%% or \MSC[2008] code \sep code (2000 is the default)
\MSC[2010] 65N30 \sep %    Finite elements, Rayleigh-Ritz and Galerkin methods, finite methods
\MSC[2010] 35J05  %    Laplacian operator, reduced wave equation (Helmholtz equation), Poisson equation

\end{keyword}

\end{frontmatter}

%% \linenumbers

%% main text
\section{Introduction}
\label{sec:introduction}

Large scale mathematical models of the groundwater flow have to deal with the presence of small scale features like wells 
and fractures that have a significant impact on the whole solution. The standard finite element method can 
capture these features using $h$ and/or $p$ adaptivity techniques 
which is payed of by a~significantly larger number of degrees of freedom. One possible alternative is the usage of 
a suitable partition of unity method (PUM) also known as an extended finite element method (XFEM). The idea is to 
augment the basis $\{\phi_n\}$ of the discrete finite element space with the functions $u_s \phi_n$, where $u_s$ is 
an a~priori known solution in the vicinity of the small scale feature. 

In this work, we use PUM on a steady two-dimensional aquifer model containing hydro-geological wells which
cause singularities in the solution. We follow the articles \cite{gracie_modelling_2010, craig_using_2011} due to Gracie and Craig,
which are, up to our best knowledge, the first works using the XFEM on the well problems. 
Recently, a~work of Ladubec, Gracie and Craig \cite{ladubec_extended_2014} was published, 
where a~time dependent two phase problem of $\rm{CO}_2$ sequestration is solved with usage of XFEM approach. 
A numerical example with a~geometry setting analogous to our single aquifer test case is presented there.

Our primary aim is to compare different partition of unity methods on a similar model. In particular, we use the XFEM 
and its corrected version (including ramp function and shift), by Fries e.g. in~\cite{fries_corrected_2008}, 
and the SGFEM introduced by Babu{\v s}ka and Banerjee in \cite{babuska_stable_2012, gupta_stable_2013}. We measure 
the convergence of the pressure head in $L^2$ norm over the aquifer domain and we compare the convergence rate of used methods. 
In contrast to the classical finite element method, the XFEM has to cautiously employ a proper integration scheme
for the assembly of the linear system on the enriched elements in order to keep the integration error small enough.
We derive an estimate for the quadrature error and propose a robust adaptive strategy based on the a priori knowledge 
of the character of the integrand.
In addition, we suggest a better choice of the enriched domain based on a tolerance criterion.

Our implementation\footnote{\url{https://github.com/Paulie14/xfem_project}} of the compared methods is done in C++ language 
using the Deal II~\cite{bangerth_deal.ii_2007}, the finite element library
not supporting any enrichment techniques at the moment. 

The paper is organized as follows. The model and its weak formulation are introduced in Section \ref{sec:model}.
In Section \ref{sec:discretization}, the discretization using different partition of unity methods is presented in detail.
An analysis of the quadrature error and rules for a robust adaptive strategy are derived in Section \ref{sec:integration}.
Section \ref{sec:enrichemnt_radius} discusses the optimal choice of the enriched domain.
Section \ref{sec:results} specifies data of the test problem and discusses numerical results,
in particular validation of convergence, behavior of the condition number, a set of experiments validating some of the theoretical results and convergence tests.
Finally, conclusions and open questions are summarized in Section \ref{sec:summary}.

\section{Model}
\label{sec:model}
We consider a steady groundwater 
flow in a system of aquifers (2D models of horizontal geological layers) separated by aquitards. 
In contrast to Gracie and Craig \cite{gracie_modelling_2010}, we suppose the aquitards to be impermeable, 
however we add an artificial volume source term. This allows us to better study the impact of 
the prescribed source on the solution which is better suited to the numerical experiments we are focused on. 

The aquifers are then connected only through the wells 
which act as sources or sinks in the domain of each aquifer. The pressure in the aquifers is further governed 
by a Dirichlet boundary condition on the outer boundary of every aquifer.

The model is defined as a complex multi-aquifer system to follow our implementation and to see the differences
we made in comparison to Gracie and Craig. Although later, we consider mostly just single aquifer.

Let $\Theta^m\subset \R^2$ be the domain of the $m$-th aquifer, $m=1,\ldots M$.
The well $w\in\mathcal{W}=\{1,\ldots,W\}$ is represented by an infinite vertical cylinder $B_w$
with center $\vc{x}_w$ and radius $\rho_w$.  We further denote 
\[
 B^m_w = B_w \cap \Theta^m, \quad \text{and} \quad
 B^m=\bigcup_{w\in \mathcal{W}}B^m_w,
\]
for any aquifer $m$ and a well $w$.
The actual computational domain of the aquifer $m$ is $\Omega^m = \Theta^m\setminus B^m$. The boundary $\partial\Omega^m$ of 
the domain consists of the exterior part $\partial\Theta^m=\Gamma^m_D$ and the interior part $\partial B^m$.


Combining the Darcy law and the continuity equation for incompressible fluid, we get
a Poisson equation for the pressure head in the $m$-th aquifer:
\begin{equation} \label{eqn:poisson}
\nabla\cdot(-\mathbf{T}^m\nabla h^m) = f^m \qquad \textrm{on } \Omega^m\subset\R^2,\; \forall m=1,\dots,M, \\
\end{equation}
which has to be supplied with boundary conditions
\begin{align}
h^m|_{\Gamma^m_D} &= h^m_D, \\
\label{eq:interior_bc}
\left(-\mathbf{T}^m\nabla h^m\cdot\vc{n}\right)|_{\partial B^m_w} &= \sigma^m_w(h^m - H^m_w) \qquad \forall w\in\mathcal{W},
\end{align}
%
% \noteJB{Na vrtu by mel byt predepsany konstantni tok dany prumerem tlaku, tj.pomoci $Q_w^m$. 
% Pak se nam objevi prumery i v nekterych dalsich rovnicich. Jde, jde o to s jakou implementaci to bylo pocitane. To bych tam nechal bez ohledu na to ze
% spravne je nektera z moznych formulaci s prumerama.
% }
where $\mathbf{T}^m\, [\textrm{m}^2\textrm{s}^{-1}]$ denotes the transmissivity tensor,
%(we will further consider only scalar $T^m$ for simplicity), 
$h^m\, [\textrm{m}]$ is the pressure head, $f^m\, [\textrm{m}\textrm{s}^{-1}]$ stands for the source density,
$\vc{n}$ is the unit outer normal vector of the interior boundary (i.e. pointing to the centers of wells),
$\sigma^m_w\, [\textrm{m}\textrm{s}^{-1}]$ denotes the permeability coefficient between $w$-th well and 
$m$-th aquifer, and finally $H_w^m$ is the pressure head in the well $w$ at the level of $m$-th aquifer.
%
\begin{figure}[!htb]
  %\vspace{-15pt}
  %TODO: add arrow (flow) to the top
  \begin{center}         
    \def\svgwidth{0.5\textwidth}
    \input{\figpath well_communcation.pdf_tex}
  \end{center}
  \caption{Flow balance in the well.}
  \label{fig:well_flows}
\end{figure}
%
The total flow from the well $w$ to aquitard $m$,
\[
    Q^m_w =-\int_{\prtl B^m_w} \sigma^m_w(h^m - H^m_w)  \dd\bx
\]
satisfies a simple balance equation on the well
\begin{align}
    \label{eq:well_flow}
    Q_w^m = Q^m_{w,in} - Q^m_{w,out} = c^{m+1}_w\left( H^{m+1}_w-H^m_w \right) - c_w^m\left( H^m_w-H_w^{m-1}\right),&\\
    \notag
    \forall\,m=1,\dots,M\text{ and }\forall\,w\in\mathcal{W},&
\end{align}
where $Q^m_{w,in}$ is the flow from the upper aquifer $m+1$, $Q^m_{w,out}$ is the flow to the lower aquifer $m-1$, and 
$c^m_w\, [\textrm{m}^2\textrm{s}^{-1}]$ is the permeability of the well $w$ in the segment below the aquifer $m$.

%
In \eqref{eq:well_flow}, we assume Darcy flow in the well for the simplicity. The bottom of the well $w$ is impermeable,
we set $c^1_w=0$, $H^0_w=H^1_w$ there, and we prescribe given pressure $H^{M+1}_w$ at the top.



%The problem is now to find set of functions $h^m\in C^2(\Omega^m)\cap{C}^1(\bar\Omega^m)$ and set of pressure 
%values in the wells $H^m_w\in\R$, for all $m\in\{1,\ldots,M\}$ and $w\in\mathcal{W}$ that
%satisfy the equations \eqref{eqn:poisson} and \eqref{eqn:well_balance}.

%Note that if the lower end of well is considered isolated from below, the coefficient must be set $c^1_w = 0$.
%The values of pressure head at the top of the wells $H^{M+1}_w$ are to be set on input. It is also possible
%to formulate additional equations
%\begin{equation} \label{eqn:well_balance_top}
%  c^M_w\left( H^{M}_w-H^{M+1}_w \right) = 0,
%\end{equation}
%to be able to set $H^{M+1}_w$ as unknowns, but we shall leave it simple at the moment.
%With $c^m_w = 0$ we can also simulate the end of the well $w$ at the level of aquifer $m$.

% We mention yet the equation \eqref{eqn:well_balance} for $m=M+1$ which is one level above the up most aquifer
% and is adjusted to the form
% \begin{equation} \label{eqn:well_balance_top}
%   c^M_w\left( H^{M}_w-H^{M+1}_w \right) = 0.
% \end{equation}
% The pressure at the top of the well $H^{M+1}_w$ can be set as an input value or, if not set, is gained 
% as part of the solution from the equation \eqref{eqn:well_balance_top}, an example of $H^4$ in \fig{fig:well_flows}.

% The boundary term in \eqref{eqn:well_balance} with large $\sigma^m_w$ would force the pressure head along the 
% well edge to be constant (equal $H^m_w$), which cannot be in general satisfied. Therefore it is weakened and 
% later replaced by the average
% \begin{equation} \label{eqn:average}
%   \avg{h^m} = \frac{1}{\abs{\partial B^m_w}} \int\limits_{\partial B^m_w} h^m \dd s,
% \end{equation}
% which corresponds to \cite{gracie}. We use 200 points around the well edge for averaging but even smaller 
% amount is possible since our test cases are symmetric.

\subsection{Weak formulation}
We define the trial space $V$ and the test space $V_0$:
\begin{eqnarray} \label{eqn:spaces}
  V &=& \left(H^1(\Omega^m)\right)^M\times\R^{W(M+1)}, \\
  V_0 &=& \left(H^1_0(\Omega^m)\right)^M\times\R^{WM},
\end{eqnarray}
where $H^1(\Omega^m)$ is the standard Sobolev space and 
\[ H^1_0(\Omega^m)=\{\varphi\in H^1(\Omega^m); \varphi|_{\Gamma^m_D}=0\}. \]
We can now introduce the weak solution $u$ and the test function $v$
\begin{eqnarray} \label{eqn:solution}
   u &=& (h^1,\ldots, h^M, H^1_1,\ldots,H^{M+1}_W)\in V, \\
   v &=& (\varphi^1,\ldots, \varphi^M, \Phi^1_1,\ldots,\Phi^M_W)\in V_0.
\end{eqnarray}
We understand $V_0$ as a subspace of $V$ setting $\Phi^{M+1}_W=0$.

To obtain the weak form, we apply the standard Galerkin method. We multiply the equation \eqref{eqn:poisson} 
by a test function $\varphi^m$ and integrate by parts over $\Omega^m$, for all $m=1,\ldots,M$, to get
\begin{equation} \label{eqn:weak_form1}
  \int_{\Omega^m} T^m \nabla h^m \cdot \nabla \varphi^m \dd\bx
  + \sum_{w\in \mathcal{W}} \int_{\partial B^m_w} \sigma^m_w (h^m - H_w^m) \varphi^m \dd\bx
  = \int_{\Omega^m} f^m\varphi^m \dd\bx.
  % - \int \limits_{\Omega^m} T^m \nabla h^m_D \nabla v^m \dd\mathbf{x},
\end{equation}
% We then multiply \eqref{eqn:well_balance} by $\Phi^m_w$, subtract it from \eqref{eqn:weak_form1} 
% which results in
% \begin{multline} \label{eqn:weak_form}
%   \int \limits_{\Omega^m} T^m \nabla h^m \cdot \nabla \varphi^m \dd\bx
%         + \sum \limits_{w\in \mathcal{W}} \sigma^m_w\left( \avg{h^m}-H^m_w\right)
%           \left(\avg{\varphi^m}-\Phi^m_w\right) + \\
%           + \sum\limits_{w\in\mathcal{W}} \Big[
%           c_w^{m+1}\left( H^m_w-H_w^{m+1}\right)\Phi^m_w - c^m_w\left( H^{m-1}_w-H^m_w \right)\Phi^m_w \Big]= \\
%   =\int \limits_{\Omega^m} f^m\varphi^m \dd\bx \quad \forall m=1,\ldots,M.
% \end{multline}
% \noteJB{Consider sum over aquifers to get square term from communication on wells.
% Boundary conditions on wells?}
We then multiply \eqref{eq:well_flow} by $\Phi^m_w$, add it to \eqref{eqn:weak_form1} 
and sum up over $m$ which results in
\begin{multline} \label{eqn:weak_form}
  \sum_{m=1}^M \; \int_{\Omega^m} T^m \nabla h^m \cdot \nabla \varphi^m \dd\bx
        + \sum_{m=1}^M \sum_{w\in \mathcal{W}} \; 
           \int_{\partial B^m_w} \sigma^m_w\left(h^m-H^m_w\right)\left(\varphi^m-\Phi^m_w\right) \dd\bx \\
        + \sum_{m=1}^{M+1} \sum_{w\in\mathcal{W}}
          c_w^{m}\left( H^{m}_w-H_w^{m-1}\right)\left(\Phi^{m}_w - \Phi^{m-1}_w\right)           
  = \sum_{m=1}^M \; \int_{\Omega^m} f^m\varphi^m \dd\bx.   
\end{multline}
% \noteJB{Consider sum over aquifers to get square term from communication on wells.
% Boundary conditions on wells?}
We say that $u\in V$ is a weak solution if it satisfies \eqref{eqn:weak_form} for all $v\in V_0$. 
Putting $h^m=\varphi^m$ and $H^m_w=\Phi^m_w$, we can clearly see the elliptic character of all terms on the 
left hand side of the weak form \eqref{eqn:weak_form} in the whole test space $V_0$. The existence and uniqueness of the solution can be shown 
via the Lax-Milgram lemma due to ellipticity and boundedness of the problem.


% Denoting $a^m(\cdot, \cdot)$ a bilinear form 
% \begin{equation} \label{eqn:bilinear_form_a}
%   a^m(u,v) = \int \limits_{\Omega^m} T^m \nabla u \cdot \nabla v \dd\bx
%         + \sum \limits_{w\in \mathcal{W}} \int \limits_{B^m_w} \sigma^m_w u v \dd\bx,
% \end{equation}
% we can write the weak form of \eqref{eqn:poisson}
% \begin{equation}% \label{eqn:weak_form}
%   a^m(h^m,v^m) - \int_{B_w^m}\sigma_w^m H_w^m v^m \dd\bx
%   = \int \limits_{\Omega^m} f^mv^m% - \int \limits_{\Omega^m} T^m \nabla h^m_D \nabla v^m \dd\mathbf{x},
%   \quad \forall m=1,\ldots,M,
% \end{equation}
% where the functions $h^m$ and test functions $v^m$ are from standard Sobolev spaces $H^1_0(\Omega^m)$,
% considering $h^m|_{\partial\Omega^m}=0$ for simplicity.
% The unknowns $H^m_w$ in \eqref{eqn:well_balance} are constants thus we can imagine that the test functions of 
% $H^m_w$ are also constants, chosen to be equal one, and so \eqref{eqn:well_balance} is not modified.

\section{Discretization}
\label{sec:discretization}
We can now proceed to the choice of the enrichment and discretization of the equation \eqref{eqn:weak_form}.
In this section as well as in the rest of the paper, we shall consider only one aquifer for the sake of simplicity
so we can omit the upper index $m$. Extension to the multi-aquifer system is straightforward. 
In particular, the space of shape functions is same on every aquifer since we consider 
the same triangulation for every aquifer in our implementation. 
Namely, we use regular square grid.

\subsection{Enrichment function}
The enrichment function can be obtained from the solution of a local problem on the neighborhood of the well $w$.
Let $\Omega_w$ be an annulus with center $\vc x_w$ inner radius $\rho_w$ and arbitrary outer radius $D \gg \rho$.
Solution of the Laplace equation $-T \Delta h = 0$ on $\Omega_w$ with any radially symmetric boundary conditions has a form
%
\begin{equation} \label{eqn:solution_form}
  h = a \log(r_w)+b, %\quad \textrm{where }
\end{equation}
where $r_w$ is a distance function
\begin{equation} \label{eqn:distance}
r_w(\vc{x}) = \|\bx - \vc{x}_w\|= \sqrt{(x-x_w)^2+(y-y_w)^2}.
\end{equation}
%
Thus the pressure head would go to infinity while closing to the center of the well.
Keeping in mind the radius of the well $\rho_w$ and the local solution \eqref{eqn:solution_form}, 
we introduce a (global) enrichment function
%
\begin{equation}
\label{eqn:enrich_func}
s_w(\bx) = 
  \begin{cases}
  \log(r_w(\bx)) & r_w > \rho_w,\\
  \log(\rho_w) & r_w \le \rho_w.\\
  \end{cases}
\end{equation}
See \fig{fig:enrich_func}.
It is natural to use the same $s_w$ on each aquifer since it depends only on $r_w$ and the wells have constant center and radius along the $x$-axis
(we consider only vertical wells, perpendicular to aquifers).
%
% and its gradient
% \begin{equation} \label{eqn:xgrad_func}
% \nabla s_w(\bx) = 
%   \begin{cases}  
%     \frac{\bx - \bx_w}{r_w^2(\bx)} & r_w>R_w \\
%     0 & r_w \leq R_w
%   \end{cases}.
% \end{equation}

%\notePE{DONE: do not forget to replace the notation of the enrichment function}
\begin{figure}[!htb]
  %\vspace{-15pt}
  %TODO: do not forget to replace the notation of the enrichment function
  \begin{center}         
    \def\svgwidth{0.5\textwidth}
    \input{\figpath enrich_func.pdf_tex}
  \end{center}
  \caption{The enrichment function.}
  \label{fig:enrich_func}
\end{figure}


In contrast to global enrichment methods, the XFEM and the SGFEM apply the enrichment functions only locally. 
Since the enrichment function is radial, it is natural to consider the enriched domain $Z_w = B_{R_w}(\vc x_w)$
of the well $w$ given by the enrichment radius $R_w$. Local enrichment methods enrich only 
nodes that are in the enriched domain of at least one well.

\subsection{Partition of unity methods}
\label{sec:pum_methods}
Let $N_\alpha(\bx)$, $\alpha\in\mathcal{I}=\{1,\ldots,N\}$ be the standard linear finite element shape 
functions associated with the node $\bx_\alpha$ of the triangulation. 
In the \textbf{standard XFEM}, we write the solution in the form
\begin{equation} \label{eqn:xfem_standard_form}
  h(\bx) = \sum_{\alpha\in\mathcal{I}}a_\alpha N_\alpha(\bx)
    + \sum_{w\in\mathcal{W}} \sum_{\alpha\in\mathcal{I}^e_w} b_{\alpha w} \phi_{\alpha w}(\bx),
\end{equation}
where $a_\alpha$ are the standard FE degrees of freedom and $b_{\alpha w}$ are the degrees of freedom coming from
the enrichment of the well $w$. The index set $\mathcal{I}^e_w$ includes all nodes enriched by the well $w$; on the other hand, 
at one node one can have several enrichment functions originating from different wells.
The local enrichment functions $\phi_{\alpha w}$ in \eqref{eqn:xfem_standard_form} are defined
in the following way
\begin{equation} \label{eqn:xfem_enrich}
    \phi_{\alpha w} = N_\alpha(\bx)L_{\alpha w}(\bx), \quad \alpha\in\mathcal{I}^e_w, w\in\mathcal{W},
\end{equation}
where the enrichment function is simply $L_{\alpha w}(\bx) = s_w(\bx)$.

\subsubsection{Corrected XFEM}
The corrected XFEM, introduced in  \cite{fries_corrected_2008}, deals with the convergence problem on blending elements
which are elements on the boundary of the enriched zone that contains both enriched and unenriched nodes.
The corrected XFEM introduces the \textbf{ramp function}
%\begin{equation} \label{eqn:ramp_function}
\begin{eqnarray} \label{eqn:ramp_function}
  G_w(\bx) &=& \sum_{\alpha\in\mathcal{I}_w^e} N_\alpha(\bx)    \\
  &=& 
  \begin{cases}
    0 & \textrm{ on unenriched elements,}    \\
    1 & \textrm{ on elements where all nodes are enriched,}    \\
    ramp & \textrm{ on elements where some of the nodes are enriched.}    \\
  \end{cases} \nonumber
%   \quad \textrm{ on } \tau, \textrm{ such that } \bx,\bx_\alpha\in\tau, \\
%   g_{\alpha w} &=&
%   \begin{cases}
%     1 & \textrm{ if } \alpha \textrm{ is enriched by } w \\
%     0 & \textrm{ otherwise. }
%   \end{cases}
\end{eqnarray}
%\end{equation}
It also extends the set of the enriched nodes of the well $w$, denoted by $\mathcal{J}^e_w$, by enriching also (previously unenriched) nodes 
of the blending elements of the well $w$. Thus $\mathcal{I}^e_w\subset\mathcal{J}^e_w$.
The enrichment function changes into the form
\begin{equation} \label{eqn:xfem_ramp}
    L_{\alpha w} = G_w(\bx) s_{w}(\bx), \quad \alpha\in\mathcal{J}^e, w\in\mathcal{W}.
\end{equation}


In the same work, i.e. \cite{fries_corrected_2008}, authors further suggest the \textbf{shifted} enrichment functions in order 
to preserve the property of the standard 
FE approximation at nodes $h(\bx_\alpha)=a_\alpha$: the value at the node is equal to the corresponding degree
of freedom. The enrichment functions must be then zero at the nodes which is satisfied in the form
\begin{equation} \label{eqn:xfem_shift}
    L_{\alpha w} = G_w(\bx) \left[s_w(\bx) - s_w(\bx_\alpha)\right],
    \quad \alpha\in\mathcal{J}^e, w\in\mathcal{W}.
\end{equation} 
The property of the shifted formulation enables us to prescribe Dirichlet boundary condition such that
$a_\alpha = h_D(\bx_\alpha)$.

It has been also shown in many cases that both ramp function and shifting are needed to obtain optimal convergence rate.
In \cite{ventura_fast_2009}, authors analyze a more general form of a ramp function (calling the method a weighted XFEM)
and compare different alternatives of shifting on crack and dislocation problems. The methods described above can be then seen
as special types of the weighted XFEM. Let us call them the \textbf{ramp function XFEM}  
and the \textbf{shifted XFEM} for the purpose of this article, as we shall reference to them later.

\subsubsection{SGFEM}
Finally, we present the \textbf{SGFEM}, according to \cite{babuska_stable_2012,gupta_stable_2013}. 
The enrichment function is defined as the subtraction of the global enrichment function and its interpolation 
\begin{equation} \label{eqn:sgfem_enrich}
    L_{\alpha w}|_{\tau} = \left[s_w(\bx) - \pi_\tau (s_w)(\bx)\right],
    \quad \alpha\in\mathcal{I}^e_w, w\in\mathcal{W}.
\end{equation} 
for any element $\tau$ of the mesh that have at least one enriched node.
The interpolation $\pi_\tau$ is built using the finite element shape functions
associated with nodes $\mathcal{I}(\tau)$ of the element $\tau$
\begin{equation} \label{eqn:sgfem_interpolation}
    \pi_\tau (s_w)(\bx) = \sum_{\beta\in\mathcal{I}(\tau)} s_w(\bx_\beta) N_\beta(\bx).
    %\quad \textrm{ on } \tau,\; \alpha\in\mathcal{I}^e, w\in\mathcal{W}.
\end{equation}
Notice that there are no additional enriched nodes on blending elements, like in $\mathcal{J}^e$ in 
\eqref{eqn:xfem_ramp} and \eqref{eqn:xfem_shift}, and no ramp function is involved.

% \subsection{Assembly}
% Having the enrichment functions defined, the enriched nodes set and the equations discretized, we can approach
% the assembly of the linear system. The system has this block pattern 
% \begin{equation}
%   \begin{pmatrix}
%   \mathbf{E}^{M+1} &               &                     & \bar{\mathbf{F}}^{M+1}   & &&&&\\
%                    & \mathbf{K}^M  & \bar{\mathbf{R}}^M  & \bar{\mathbf{C}}^M       & &&&&\\
%                    & \mathbf{R}^M  & \mathbf{S}^M        & \bar{\mathbf{D}}^M       & &\ddots&&&\\
%   \mathbf{F}^{M+1} & \mathbf{C}^M  & \mathbf{D}^M        & \mathbf{E}^M             & &&&&\\
%   &&&& \ddots &&&& \\
%   &&&& & \mathbf{E}^2 &              &                    & \bar{\mathbf{F}}^2 \\
%   &&&\ddots& &              & \mathbf{K}^1 & \bar{\mathbf{R}}^1 & \bar{\mathbf{C}}^1 \\
%   &&&& &              & \mathbf{R}^1 & \mathbf{S}^1       & \bar{\mathbf{D}}^1 \\
%   &&&& & \mathbf{F}^2 & \mathbf{C}^1 & \mathbf{D}^1       & \mathbf{E}^1 \\
%   \end{pmatrix}
%   \begin{pmatrix}
%     \mathbf{H}^{M+1} \\
%     \mathbf{a}^{M} \\
%     \mathbf{b}^{M} \\
%     \mathbf{H}^{M} \\
%     \mathbf{\vdots} \\
%     \mathbf{H}^{2} \\
%     \mathbf{a}^{1} \\
%     \mathbf{b}^{1} \\
%     \mathbf{H}^{1} \\
%   \end{pmatrix} =
%   \begin{pmatrix}
%     \mathbf{f}^{M+1}_H \\
%     \mathbf{f}^{M}_a \\
%     \mathbf{f}^{M}_b \\
%     \mathbf{f}^{M}_H \\
%     \mathbf{\vdots} \\
%     \mathbf{f}^{2}_H \\
%     \mathbf{f}^{1}_a \\
%     \mathbf{f}^{1}_b \\
%     \mathbf{f}^{1}_H \\
%   \end{pmatrix}
% \end{equation}
% where $\mathbf{K}^m$ is the matrix of the standard FE approximation, $\mathbf{S}^m$ is the matrix of the 
% enrichment, $\mathbf{R}^m$ relates standard FE and enrichment,  $\mathbf{E}^m$ is the matrix associated with 
% unknown pressures in the wells and 
% $\mathbf{C}^m$, $\mathbf{D}^m$ and $\mathbf{F}^m$ are communication matrices: standard FE -- well, 
% enrichment -- well and well -- well, respectively. The whole system is symmetric.
% 
% The entries of the matrices are 
% 
% \begin{eqnarray}
%   K^m &=& \left[a^m(N^m_\alpha, N^m_\beta)\right]   ,\qquad \forall \alpha,\beta\in \mathcal{I} \label{eqn:k_entry}\\
%   S^m &=& \left[a^m(\phi^m_{\alpha k}, \phi^m_{\beta j})\right]     ,\qquad \forall \alpha,\beta\in \mathcal{I}^e,\; \forall k,j\in\mathcal{W} \label{eqn:s_entry}\\
%   R^m &=& \left[a^m(N^m_{\alpha}, \phi^m_{\beta j})\right]      ,\qquad \forall \alpha\in \mathcal{I},\; \forall\beta\in\mathcal{I}^e,\; \forall j\in\mathcal{W} \label{eqn:r_entry}\\
%   C^m_{w\alpha} &=&-\int\limits_{B^m_w} \sigma^m_w N^m_\alpha   ,\qquad \forall \alpha\in \mathcal{I},\; \forall w\in\mathcal{W}\\
%   D^m_{w\alpha j} &=&-\int\limits_{B^m_w} \sigma^m_w \phi^m_{\alpha j}  ,\qquad \forall \alpha\in \mathcal{I}^e,\; \forall w\in\mathcal{W}\\
%   \rm{diag}(\mathbf{E}^m)_w &=&\int\limits_{B^m_w} \sigma^m_w + c^{m+1}_w   ,\qquad \forall w\in\mathcal{W}\\
%   \rm{diag}(\mathbf{F}^m)_w &=& -c^m_w ,\qquad \forall w\in\mathcal{W}
% \end{eqnarray}


\section{Integration on enriched elements}
\label{sec:integration}
In order to compute the entries of the system matrix, %\eqref{eqn:s_entry} and \eqref{eqn:r_entry} 
we need to integrate
the expressions containing the enrichment functions. These of course can be non-polynomial, like they are 
in our case. The standard quadrature rules are not appropriate any more, for they are constructed to integrate 
precisely only polynomials up to a given degree. The higher requirements on the integration precision
are the price for using enrichment functions and a coarse mesh.

There are two aspects which the integration must handle properly:
\begin{itemize}
  \item the steep gradient of enrichment base functions in the vicinity of the well (the singularity),
  \item the well edge geometry, since we need to integrate only outside of the well.
\end{itemize}

One of the approaches to deal with these requirements is an adaptive quadrature. The element is divided into 
subregions (squares on the reference element) only to place more quadrature points inside but not to bring 
any more degrees of freedom into the system. In this section we will discuss the adaptivity rules, 
suggest an improvement and compare our strategy to the original one (developed in \cite{gracie_modelling_2010}).

\subsection{Instability of adaptive quadrature}
\label{sec:refinement_element}
Gracie and Craig in \cite{gracie_modelling_2010} refine only subregions that cross the boundary of the well, using at most 12 refinements.
This catches nicely the well edge but it works only when the well is placed at the node of an element or near the center of an element. 
When the well is placed near the edge of an element, there can be
a large difference in the size of neighboring subregions, see \fig{fig:adapt_ref_a}. Although
the integrand is computed precisely enough on the element with the well inside, the quadrature points on the
neighboring elements (where the integrand has still large derivatives) are placed very sparsely 
and the integration error is large.

\begin{figure}[!htb]
%   \vspace{0pt}
  \centering    
  \subfloat[refinement due to Gracie and Craig]{\label{fig:adapt_ref_a} 
    \includegraphics[width=0.45\textwidth]{results/adaptive_refinement_3_old.pdf} }
  \hspace{0pt}
  \subfloat[improved refinement]{\label{fig:adapt_ref_b} 
    \includegraphics[width=0.45\textwidth]{results/adaptive_refinement_3_new.pdf} }
  \caption[Adaptive refinement comparison]
  {Comparison of the original and improved refinement techniques.
   Black lines denote enriched elements edges, red lines denote adaptive refinement (subregions edges) and the well
   edge is blue.
  }
  \label{fig:adapt_refinement}
\end{figure}
In order to overcome this instability of the adaptive quadrature, we have made an asymptotic analysis of the integration error presented 
in the next section.

\subsection{Estimate of quadrature error}

Let us assume only one well of radius $\rho$ situated at the origin. In the case of elliptic equation, the term with the strongest singularity is 
\begin{equation}
    \label{eq:term-of-interest}
    f(r)=(\nabla \log r )^2 \approx r^{-2}
\end{equation}
which is also the worst term to integrate independently of the particular variant of the PU method.
Consider a~square subregion $S$ with a~side $\delta$ and 
let us denote $r_{S}$ its distance from the origin.
We want to estimate the error of the 2D tensor product Gauss quadrature rule of order $n$ ($n$ times $n$ points) on the square $S$. 
Let us denote 
$\Pi^n f$ the projection of the integrand $f$ to the space of polynomials that are integrated exactly.
We were not able to find error estimates for 2D quadratures in the literature and deriving them would be extremely technical.
However, we can make an observation that among the squares of the same $r_{S}$, the quadrature error is the highest for the squares laying on one of the axis.
Assume without loose of generality a square on the $X$-axis. Since $\abs{y}<\delta \ll \abs{x}$, the monomials of $\Pi^n f$ containing $y$ 
are negligible and we get the quadrature
of order $n$ in the radial ($X$-axis) direction. On the other hand for the square on the diagonal, the bilinear terms of 
a 2D quadrature effectively enhance its order to $2n$ in the radial (diagonal) direction. Due to the radial nature of the integrand,
we can estimate the quadrature error on the square $S$ by the error on the square $S'$ with the same $r_S$ laying on the $X$-axis, then we can 
neglect $y$~monomials and use the error estimates for the 1D quadrature:
\[
  \int_S \abs{f-\Pi^n f} \dd\bx \le \int_{S'}\abs{f-\Pi^n f} \dd\bx \le \delta E^n((r_{S}, r_{S}+\delta)).
\]
The $E_n$ is the error of 1D Gauss quadrature of the order $n$ ($n$ quadrature points) over the interval $(r,r+\delta)$
\[
  E_n = \frac{\delta^{2n+1} (n!)^4}{(2n+1)((2n)!)^3} f^{(2n)}(\xi_n) 
\]
for some $\xi_n \in (r, r+\delta)$, see e.g. \cite{kahaner_numerical_1989}. 
The expression $f^{(2n)}$ denotes a derivative of order $2n$ of a function $f$.
Regarding the integrand \eqref{eq:term-of-interest}, we have 
\[
  \abs{f^{(2n)}(r)} = (2n+2)! r^{-(2n+2)}.
\]
Finally, we get an estimate for the quadrature error on a single subregion:
\[
    \int_S \abs{f-\Pi^n f}  \dd\bx \le  \alpha_n \left( \frac{\delta}{r_{S}} \right)^{2n+2}, 
  \qquad \alpha_n = (2n+2)\left( \frac{(n!)^2}{(2n)!} \right)^2.
\]
This estimate implies that we have to ensure $\delta < r_S$ in order to get a decent quadrature error 
and possibly to employ a higher order quadrature. 


%JB%Derived criterion holds only on squares, where the integrated function is smooth.
%JB%This is not the case for squares intersecting the boundary of the well, $r_{S} \le \rho \le r_{max}$, where we integrate 
%JB%discontinuous function $\chi_{S \setminus W} f$. Using substitution, we can map $W$ to unit circle $B$
%JB%\[
%JB%  \int_{S} \chi_{S \setminus W} \frac{1}{r^2} \d \vc r= \int_{S'} \chi_{S' \setminus B} \frac{1}{r'^2} \d \vc {r'},
%JB%\]
%JB%where $S'$ is square with side $H=\delta/\rho$ and $\vc {r'} = \vc{r}/\rho$. Empirically determined error of the midpoint rule for later 
%JB%integral is
%JB%\[
%JB%    E(H) = c_e H^{p_e}, \quad \text{with } c_e=0.08,\ p_e=2.5.
%JB%\]
%JB%This error has to be smaller then $\epsilon h^2$, thus for $h$ we get formula:
%JB%\begin{equation} \label{eqn:h_criterion}
%JB%   h\le h_b(\epsilon) = \Big(\frac{\epsilon \rho^{p_e}}{c_e}\Big)^{\frac{1}{p_e-2}}. 
%JB%\end{equation}
%JB%
%\noteJB{TODO: modify test of integration on unit disk for the function $1/r^2$.}

\subsection{A priori adaptive quadrature rules}
Let us denote $r_{min}$ the minimum and $r_{max}$ the maximum distance to the center of a well from a subregion. 
Based on the analysis presented above, we propose following adaptive quadrature rules:
%JB%\begin{enumerate}
%JB% \item If $r_{max} < \rho$ the square quadrature is zero.
%JB% \item If $r_{min} < \rho < r_{max}$ (square cross the well boundary) we subdivide the square unless $\delta < 2^{-12}h$.
%JB% \item If $r_{min} > \rho$. For $\frac{h}{r_{min}} > \frac{1}{2}$ subdivide the square, else select order $n$ so that 
%JB% \begin{equation} \label{eqn:alpha_criterion}
%JB%    \frac{\alpha_n h^{2n}}{r_{min}^{2n+2}} \le \epsilon,
%JB% \end{equation}
%JB% use at least the same order as necessary for FEM.
%JB%\end{enumerate}
%JB%
%\noteJB{TODO:
%We should estimate true error numerically using one more subdivision and compare it to prescribed tolerance, we should be safely 
%below, without increasing the number of evaluation compared to current implementation.
%}



%JB%We suggest additional criterion for subelements refinement which takes into account a subelement diameter 
%JB%and its distance from the well
%JB%\begin{equation}
%JB%  h \leq C_R r_{min},
%JB%\end{equation}
%JB%\notePE{Originally, I compute $r_{min}$ between vertices and the well center, but I believe that the results would be the same.}
%JB%where $h$ is the diameter of the subelement and $r_{min}$ is the minimal distance between a vertex of 
%JB%the subelement and the well center. $C_R$ is a scaling constant, equal 0.5 by default, through which we can 
%JB%control the significance of the criterion. If satisfied, the subelement is not refined anymore.
%JB%

%JB%Eventually, we do 10 levels of improved adaptive refinement with the following rules ($r_{max}$ is the maximal
%JB%distance of a vertex of the subelement and the well center):
\begin{enumerate}
 \item If $r_{max} < \rho$, the subregion is not refined and the quadrature is zero.
 \item If $r_{min} < \rho < r_{max}$ and $\delta > 2^{-10}h$, the subregion is refined.
 \item If $r_{min} < \rho < r_{max}$ and $\delta \le 2^{-10}h$, $3\times3$ Gauss quadrature is used.
 The values at quadrature points lying inside the well are set to zero.
 \item If $r_{min} > \rho$ and $\delta > r_{min} / 2$, the subregion is refined.
 \item If $r_{min} > \rho$ and $\delta \le r_{min} / 2$, $3\times3$ Gauss quadrature is used.
\end{enumerate}


These rules ensure $\delta < r_{min}/2$ outside the well, where the integrand is smooth. Subregions intersecting 
the well's boundary are refined using at most $10$ refinement levels, since the integrand is discontinuous there and we cannot employ 
estimates from the previous section. The maximum number of levels is chosen so that we get the similar total number of quadrature points 
as in the quadrature used by Gracie and Craig in \cite{gracie_modelling_2010}. Using the proposed rules, the elements that do not contain the well are refined as well,
see \fig{fig:adapt_ref_b}. 

Implementation of this approach allows the convergence rates of the used PU methods to close up to the optimum,
as the results of the numerical tests will show in section \ref{sec:results}.


% In \fig{fig:adapt_refinement_norm} you can see the $L_2$ norms of the error on the enriched elements.
% The first one is computed using the original adaptive quadrature where $3\times3$ quadrature rule is applied 
% on the enriched element with the well and $4\times4$ quadrature rules on the other enriched elements. 
% The improved adaptive quadrature is computed as described above. Notice the scale of the improved version -- 
% the error on elements is in small range and is not significantly concentrated anywhere. In contrast, 
% the original version shows out large error that is concentrated on the closest non-refined element to the well.
% 
% 
% \begin{figure}[!htb]
% %   \vspace{0pt}
%   \centering    
%   \subfloat[original]{\label{fig:adapt_ref_norm_a} 
%     \includegraphics[width=0.45\textwidth]{results/adaptive_refinement_extract_3_old.pdf} }
%   \hspace{0pt}
%   \subfloat[improved]{\label{fig:adapt_ref_norm_b} 
%     \includegraphics[width=0.45\textwidth]{results/adaptive_refinement_extract_3_new.pdf} }
%   \caption[Adaptive quadrature refinement comparison]
%   {Comparison of element-wise error in $L_2$ norm using original and improved adaptive quadrature.
%   \notePE{TODO: add axis, remove legend label, try to sharpen}
%   }
%   \label{fig:adapt_refinement_norm}
% \end{figure}

% \subsection{Circle integration experiment}
% The approximation of the integration domain is also the source of the error. We decided to run 
% an experiment on integrating the characteristic function of the well with our adaptive integration.
% The domain $\Omega$ is a square $4\times4$ out of which a circle of radius 1.0 is cut off. The characteristic 
% function is considered
% \begin{equation}
%   \chi(\bx) = \left\{
%     \begin{array}{l l}
%       0 & \quad \textrm{if } \bx \textrm{ is inside the circle}\\
%       1 & \quad \textrm{otherwise}
%   \end{array} \right.
% \end{equation}
% and the integral 
% \begin{equation}
%   \int_{\Omega}\chi(\bx) \dd\bx = 4^2 - \pi
% \end{equation}
% is equal the area of the square minus the area of the circle.
% 
% In this experiment we investigate the influence of the order of the quadrature rule and the level of
% the refinement on how precisely the well geometry is captured.
% 
% In the graph in \fig{fig:adapt_ref_convergence} we can see that for all the quadratures the convergence rate
% is similar, around 1.5. The gain from using higher order quadratures is not worth, especially in case 
% of the order 4 the error is not much smaller than the error of the quadrature of the order 3. 
% 
% Finally the highest level of refinement is chosen to be 10 and the quadrature order to be 3. The number of
% the quadrature points generated by the process described above in \ref{sec:refinement_element} is then similar 
% both in the original (14793) and improved version (14819).
% 
% \begin{figure}[!htb]
% %   \vspace{0pt}
%   \centering    
%   \includegraphics[width=0.7\textwidth]{results/adaptive_integration.pdf}
% %   \subfloat[rozdìlený element s vrtem]{\label{fig:adapt_ref_a} 
% %     \includegraphics[width=70mm]{\figpath adaptive_ref.pdf} }
% %   \hspace{0pt}
% %   \subfloat[detail hranice vrtu]{\label{fig:adapt_ref_b} 
% %     \includegraphics[width=72mm]{\figpath adaptive_ref_detail.pdf} }
%   \caption[Adaptive refinement convergence]{Convergence of adaptive refinement of a circle cutoff.}
%   \label{fig:adapt_ref_convergence}
% \end{figure}



%JB%\subsection{Adaptive integration experiment}
%JB%We now describe the experiment from which we obtained the coefficients in the criterion 
%JB%\eqref{eqn:h_criterion}. Let us have a single element, a square $4\times4$, out of which a circle of unit radius 
%JB%is cut off -- this represents an element with a well. We now want to investigate integration error
%JB%in the vicinity of the circle on the function $f=r^{-2}$, $r$ being the distance from the center of the circle. 
%JB%
%JB%We compute the integral on the selected level of refinement only on the squares intersecting the circle. Then
%JB%we refine the squares up to the 12-th level, integrate again and compute the difference. Quadratures rules
%JB%$1\times1$, $2\times2$, $3\times3$ and $4\times4$ are used. The results are shown in 
%JB%\fig{fig:adapt_integration_conv} also with convergence trend lines equations. The conclusion can be made
%JB%that it is more efficient to refine one more level with one-point quadrature than to use higher order 
%JB%quadrature on the coarser level. As for the coefficient, these can be read from the graph -- for the one point
%JB%quadrature $c_e=12.65$ and $p_e=1.27$.
%JB%
%JB%\begin{figure}[!htb]
%JB%%   \vspace{0pt}
%JB%  \centering    
%JB%  \includegraphics[width=0.9\textwidth]{results/adapt_integration_conv.pdf}
%JB%%   \subfloat[rozdìlený element s vrtem]{\label{fig:adapt_ref_a} 
%JB%%     \includegraphics[width=70mm]{\figpath adaptive_ref.pdf} }
%JB%%   \hspace{0pt}
%JB%%   \subfloat[detail hranice vrtu]{\label{fig:adapt_ref_b} 
%JB%%     \includegraphics[width=72mm]{\figpath adaptive_ref_detail.pdf} }
%JB%  \caption[Adaptive quadrature convergence.]{Convergence graph of adaptive quadrature on function $r^{-2}$.
%JB%  \\ \notePE{thicker trend lines}}
%JB%  \label{fig:adapt_integration_conv}
%JB%\end{figure}
%JB%
\section{Estimate of the enrichment radius} \label{sec:enrichemnt_radius}
In this section, we shall study the dependence of the solution error on the enrichment radius $R$. First part is devoted to 
theoretical analysis, the later part presents numerical results.
Let us consider a general elliptic problem to find $u\in V$ satisfying
\[
   a(u, \phi) = \langle f, \phi \rangle, \text{ for } \phi \in V,
\]
where $a$ is bounded elliptic bilinear form: $\norm{a}\le M_a$, $a(v, v) \ge \gamma \norm{v}_V^2$, $\gamma>0$, and $f$ a bounded linear form, $f\in V'$. 
Suppose, that the problem is to be solved on a domain $\Omega \subset \R^2$ with a single hole (well) of radius $\rho$ at the origin. 
Let us assume, that the solution can be split into the singular part $u_s(\vc x)= \log |\vc x|$ and the regular part $u_r=u-u_s$.
Let $V^P_h$ be a polynomial finite element subspace of $V$ on a regular mesh of elements with a maximal diameter $h$
and let $V_h$ be a such enriched space that $u_s$ can be approximated exactly on the enriched domain $Z_R$, i.e.
\[
   \inf_{v\in V_h} \norm{u_s - v}_V = \inf_{v\in V^P_h} \norm{u_s|_{Z'_R} - v}_V, \quad Z'_R = \Omega\setminus Z_R.
\]
Using standard error estimate for elliptic PDE (e.g. \cite[Theorem 13.1]{ciarlet_basic_1991}), we get
\begin{equation}
    \label{eq:std_err_estimate}
    \norm{u - u_h}_{V} \le c_a \inf_{v \in V_h} \norm{u - v}_{V} 
    \le c_a \big(\inf_{v \in V^P_h} \norm{u_r - v}_{V} + \inf_{v \in V_h} \norm{u_s - v}_{V} \big).   
\end{equation}
where $c_a=1+\norm{a}/\gamma$.
In the following, we consider $V=H^1(\Omega)$, square grid and $V^P_h$ formed by bilinear finite elements. 
Then \eqref{eq:std_err_estimate} can be further estimated using approximation property of $V^P_h$:
\begin{equation}
    \label{eq:particular_estimate}
    \norm{u - u_h}_{H^1(\Omega)} \le c_a \big(c h \abs{u_r}_{H^2(\Omega)} + \norm{u_s - \Pi u_s}_{H^1(Z'_R)} \big)   
\end{equation}
where $\Pi u_s$ denotes interpolation of $u_s$ in $V^P_h$. Our next aim is to find tight estimate for the second term.
To this end, we calculate $H^1$ error on a single square element $S_{h,r}$ with side $h$ and distance $r$ from origin.
Using parametrization $0<s,t<1$,  we get

\begin{align*}
 (u_s - \Pi u_s)(s,t)&=\log\sqrt{(r+hs)^2+(ht)^2} -\Big[(1-s)(1-t)\log r\\
 &\quad+ (1-s)t\log\sqrt{r^2+h^2} + s(1-t) \log(r+h) \\
 &\quad+ st\log\sqrt{(r+h)^2+h^2} \Big]\\
 &=\frac12 \frac{h^2}{r^2}\big(t^2-t - s^2 +s\big) + O(\frac{h^3}{r^3})
\end{align*}
and 
\begin{equation}
 \nabla(u_s - \Pi u_s)(s,t) = \frac{h}{r^2} \Big( \frac12-s, t-\frac12 \Big) + O(\frac{h^2}{r^2}).
\end{equation}
Assuming $h<r$, we can neglect higher order terms. Then, we obtain by direct integration
\begin{align*}
 \norm{u_s - \Pi u_s}^2_{L^2(S_{h,r})} \approx \frac14 \frac{h^6}{r^4}\int_0^1\int_0^1 (t^2-t-s^2+s)^2\,\d s\, \d t = \frac{1}{360}\frac{h^6}{r^4} 
\end{align*}
and
\begin{equation}
    \label{eq:grad_estimate_on_square}
    \norm{\nabla(u_s - \Pi u_s)}^2_{L^2(S_{h,r})} \approx \frac{2h^4}{r^4} \int_0^1 \Big(t-\frac12\Big)^2 \d t = \frac{1}{6}\frac{h^4}{r^4}.
\end{equation}
Thus for the density of squared error we have
\[
    \frac{1}{\abs{S_{h,r}}} \norm{u_s - \Pi u_s}^2_{H^1(S_{h,r})} \approx \frac{h^2}{6r^4}
\]
which after integration over the unenriched domain gives final estimate:
\begin{equation}
    \label{eq:singular_approx_error}
    \norm{u_s - \Pi u_s}_{H^1(Z'_R)} \le \left[\int_0^{2\pi} \int_R^\infty \frac{h^2} {6r^4} r \,\d r\, \d \phi\right]^{1/2} = \sqrt\frac{\pi}{6}\frac{h}{R}. 
\end{equation}

Recalling the estimate \eqref{eq:std_err_estimate}, we can conclude that optimal choice of the enrichment radius is $h/R\approx \norm{u_r-\Pi u_r}_{H^1(\Omega)}$, 
which balance error in the regular and the singular part. In combination with an a posteriori error analysis, this could give a rule for an automatic
determination of the enrichment radius.



\section{Numerical results}
\label{sec:results}

\subsection{Test problems} \label{sec:test_cases}
In this section, we define problems that we solve in our numerical experiments. We restrict ourselves to 
single aquifer problems for the purpose of this article but our implementation enables multi-aquifer systems 
as well. In order to measure convergence, it is desirable to have an analytic solution which will be also derived. 

Suppose we have the following problem
\begin{equation} \label{eqn:poisson_problem}
\left.\begin{aligned}
    -T \Delta h &= TU\omega^2\sin(\omega x) \qquad \textrm{on } \Omega, \\
    h|_{\Gamma_D} &= h_D,\\
    \left(-T\nabla h\cdot\vc{n}\right)|_{\partial B_w} &= \sigma_w\left(h - H_w\right), \\
    \end{aligned}
    \;\right\}
\end{equation}
where the pressure head $H_w$ at the well and $h_D$ on the boundary are given. 
The domain $\Omega$ is a square with a diagonal of length $2D$ with the well placed at the point $\vc{x}_w$.
Note that we neglected the influence of $c^2_w$ by eliminating the equation from the system (we consider 
$H_w = H^1_w = H^2_w$). We further set $c^1_w=0$ and $T=1.0$. 

We will look for the solution $h$ of \eqref{eqn:poisson_problem} in the form
\begin{equation} \label{eqn:poisson_solution}
  h=a\log\left(\frac{r_w}{D}\right)+U\sin(\omega x),
\end{equation}
which solves the Poisson equation in \eqref{eqn:poisson_problem}. We also set $h_d=h$.
To obtain $a$, we use the average pressure head along the well edge in the third equation 
of \eqref{eqn:poisson_problem}:
\begin{equation} \label{eqn:a_average_derivation}
     -\langle\nabla h \cdot \vc{n}\rangle = \sigma_w\left(H_w - \langle{h}\rangle \right), 
    \qquad \textrm{ where } \langle{f}\rangle = \fint_{\partial B_w} f  \dd\bx.
\end{equation}
The integral in \eqref{eqn:a_average_derivation} cannot be calculated precisely therefore it is approximated with error 
$O(\omega^2 \rho_w)$. Finally, we have
\begin{equation} \label{eqn:a_constant}
    a=\frac{\rho_w \sigma_w \big[H_w - U\sin(\omega x_w) - \frac{U\omega^2}{2\sigma_w}\sin(\omega x_w)\big]}
           {\rho_w \sigma_w \log\left(\frac{\rho_w}{D}\right) - 1}.
\end{equation}
The solution \eqref{eqn:poisson_solution} is then used to define the boundary function $h_D$.

\begin{figure}[!htb]
%   \vspace{0pt}
  \centering    
  \subfloat[geometry of both problems]{\label{fig:geometry} 
%     \begin{center}         
      \def\svgwidth{0.325\textwidth}
      \input{results/geometry.pdf_tex}
%     \end{center} 
      }
  \hspace{5pt}
  \subfloat[solution of Problem 2]{\label{fig:solution} 
    \includegraphics[width=0.58\textwidth]{results/solution_final.pdf} }
  \caption[]
  {Geometry and solution of the problem.}
%   \label{fig:adapt_refinement}
\end{figure}

Let us now define the input data. The domain $\Omega$ is a square $(-100,100)\times(-100,100)$ and the well is characterized by 
$\vc{x}_w=[5.43,5.43]$,  $\rho_w=0.2$, $H_w=100$ and $\sigma_w=10^5$, see geometry in \fig{fig:geometry}. The \prob{1} is to find solution $h$ of
\eqref{eqn:poisson_problem} with $U=0.0$ which means that the right hand side of the Poisson equation vanishes.
The \prob{2} is to find solution $h$ of \eqref{eqn:poisson_problem} with $U=8.0$ and $\omega=0.03$, see \fig{fig:solution}.


% \begin{prob}[Laplace equation] \label{def:test_case_1}
% Find the solution $h$ of a single aquifer problem
% \begin{eqnarray*} \label{eqn:laplace_problem}
%     -T \Delta h &=& 0 \qquad \textrm{on } \Omega \\
%     h|_{\partial\Omega} &=& h_D \\
%     H_w = H^2_w &=& P_w \\
%     \left(-T\nabla h\cdot\vc{n}\right)|_{\partial B_w} &=& q_w \\
% \end{eqnarray*}
% where the pressure $P_w$ at the well and the pressure $h_D$ at the boundary are given.
% The domain $\Omega$ is a square with a single well placed near the center in $\vc{x}_w$.
% \end{prob}
% We can find the analytic solution of \probref{def:test_case_1} but on a circular disk where we have the zero 
% pressure on the outer boundary. The solution is considered in the form $h=a\log(\frac{r_w}{R})$ 
% where $R$ is the radius of the circular domain. Transmisivity is further set $T=1.0$ for simplicity.
% Now we can use the boundary condition to find the constant $a$ such that it satisfies
% \begin{eqnarray*}
%   -T\nabla h \cdot \vc{n} = \sigma_w(P_w - h),
% \end{eqnarray*}
% where $\rho_w$ is the well radius.
% The solution is then
% \begin{equation} \label{eqn:laplace_solution}
%   h=a\log\left(\frac{r_w}{R}\right) \qquad \textrm{ with } 
%     a=\frac{\rho_w \sigma_w P_w}{\rho_w \sigma_w \log\left(\frac{\rho_w}{R}\right) - 1}
% \end{equation}
% The function \eqref{eqn:laplace_solution} is just the analytic solution of \probref{def:test_case_1} when 
% \eqref{eqn:laplace_solution} is used to compute $h_D$ on the square boundary, with $R$ being the half 
% diagonal of the square.
% 
% \begin{prob}[Poisson equation] \label{def:test_case_2}
% Find the solution $h$ of a single aquifer problem with a source term
% \begin{eqnarray*}% \label{eqn:poisson_problem}
%     -T \Delta h &=& TU\omega^2\sin(\omega x) \qquad \textrm{on } \Omega \\
%     h|_{\partial\Omega} &=& h_D + U\sin(\omega x)\\
%     H_w = H^2_w &=& P_w \\
%     \left(-T\nabla h\cdot\vc{n}\right)|_{\partial B_w} &=& q_w \\
% \end{eqnarray*}
% where $U$ and $\omega$ are given.
% \end{prob}
% %where $U$ is the amplitude and $\omega$ is the angular frequency 
% The analytic solution of \probref{def:test_case_2} we are looking for is in the form
% \begin{equation} \label{eqn:poisson_solution}
%   h=a\log\left(\frac{r_w}{R}\right)+U\sin(\omega x).
% \end{equation}
% Obtaining constant $a$ is now more difficult. We consider following approximation by averging of
% pressure head around the well edge:
% \begin{equation} \label{eqn:a_average_derivation}
%     \fint \limits_{\partial B_w} h = \langle{h}\rangle, \qquad
%     \fint \limits_{\partial B_w} -\nabla h \cdot \vc{n} = \sigma\left(P_w - \langle{h}\rangle \right).
% \end{equation}
% % \begin{eqnarray*} \label{eqn:a_average_derivation}
% %     \fint \limits_{\partial B_w} h &=& \langle{h}\rangle, \\
% %     \fint \limits_{\partial B_w} -\nabla h \cdot \vc{n} &=& \sigma\left(P_w - \langle{h}\rangle \right)
% % \end{eqnarray*}

% The input parameters for the numerical tests of \probref{def:test_case_1} and \probref{def:test_case_2} 
% are gathered in the table \ref{tab:parameters}.
%
% \begin{table}[!ht]
% \begin{center}
% \begin{tabular}{crr}
% \toprule
% % \multicolumn{2}{c}{Item} \\
% % \cmidrule(r){1-2}
% parameter    & value \\
% \midrule
% $\Omega$   & $(-100,100)\times(-100,100)$   \\
% $\vc{x}_w$  & $[5.43,5.43]$   \\
% transimisivity $T$          & 1.0   \\
% boundary pressure $P_D$     & 0.0   \\
% well pressure $P_w$         & 100.0 \\
% well radius $\rho_w$        & 0.2 \\
% $\sigma_w$                  & $10^5$ \\
% $c_w^0$, $c_w^1$            & 0.0, $10^{13}$ \\
% $U$                         & 8.0 \\
% $\omega$                    & 0.03 \\
% \bottomrule
% \end{tabular}
% \caption{Input parameters for the numerical tests. Units are omitted.
% \label{tab:parameters}
% \end{center}
% \end{table}
%

\subsection{Comparison of PU methods} \label{sec:res_comparison}
In this section, we present the convergence results. We solve \prob{1} and \prob{2} 
defined in section \ref{sec:test_cases} with the methods described in \ref{sec:pum_methods} and compare them.
The linear algebraic system is always solved by conjugate gradients method (CG) with Jacobi preconditioning 
and tolerance $10^{-9}$. The error of the approximation is computed in $L^2$ norm and so the convergence is
considered in this norm.

Let us start with \prob{1} and look at the convergence graph in \fig{fig:convergence}.
At first, we solve the problem by the standard FEM with the adaptive mesh refinement where 30\% of the elements
with the highest error estimate are refined (we use the Kelly's error estimator built in the Deal II library).
The element size $h$ is determined as the size of the smallest elements in the 
vicinity of the well. We see that the convergence is slow until the size of elements reaches the scale of the
well. Therefore we divide the graph in two parts with convergence orders 0.56 and 1.27.
We see that a very fine mesh is needed to capture the singularity but the error is still high.
%On the other hand the adaptive refinement can save us a lot of degrees of freedom and it is simple to deal 
%with the hanging nodes coming from the adaptive refinement. 
%Although changing the position of the well means also editing the mesh.

The standard XFEM pushes the error down by three orders of magnitude. It has the optimal convergence rate with order 2.0 but the system
matrix condition number grows rapidly and for $h<2$ the CG solver does not converge even in 10000 iterations. The same
problem rises up using the ramp function XFEM. It deals better with the error on blending elements but the
ill-conditioning of the system matrix still corrupts the computation. We discuss the conditioning of the system 
a little bit more in the next subsection \ref{sec:res_conditioning}.

The shifted XFEM and the SGFEM behave nearly the same way and give the best results. We only mention that the SGFEM saves 
small amount of degrees of freedom on blending elements in comparison with the shifted and the ramp function XFEM.
The order of convergence in $L^2$ norm closing to 2.0 is optimal.

\begin{figure}[!htb]
%   \vspace{0pt}
  \centering    
  \includegraphics[width=\textwidth]{results/convergence.pdf}
  \caption[Convergence graph \prob{1}]{Convergence of the $L^2$ norm of the error for different methods and Problem 1.}
  \label{fig:convergence}
\end{figure}


The results of the convergence test for \prob{2} are presented in \fig{fig:convergence_sin}.
For the comparison, we also plot the error of the standard FEM approximation of the problem with the well omitted (the regular part $u_r$), 
It is labeled 'FEM (ur)' since the solution corresponds to the regular part $u_r$ of the solution to \prob{2} and displays the optimal convergence order 2.0.

In the case of the standard FEM, we see the same behavior as in \fig{fig:convergence}. The dominant error still 
comes from the singular part of the solution. All the PU methods have again nearly optimal convergence rate. 
The standard XFEM and the ramp function XFEM
are stopped at element size close to 1.0 due to the growth of the condition number of the system matrix.

Regarding the order convergence 1.8 presented by Gracie and Craig \cite{gracie_modelling_2010}, we obtained similar convergence order 
around 1.7-1.8 in our experiments using the original adaptive quadrature. Although, the order could be lower 
depending on the position of the well to the nodes of the mesh. We do not experience this behaviour with our adaptive
quadrature and the convergence order is always close to the optimum of 2.0.

While decreasing the element size under 1.0, we engage some new kind of error in the shifted XFEM and the 
SGFEM solution. We believe the adaptive quadrature has nothing to do with this matter after running several tests 
-- the error is not lower when increasing the number
of refinement levels or using a~higher order quadrature. Possible reason could be the boundary condition 
on the interior boundary. In our formulation (c.f. \eqref{eq:interior_bc}), we allow a non-constant flow around the boundary
of the well, while the enrichment function assumes strictly radially symmetric flow. Using an averaged form of the boundary 
condition similar to \eqref{eqn:a_average_derivation} both in the formulation and in the implementation may give better results.

\begin{figure}[!htb]
%   \vspace{0pt}
  \centering    
  \includegraphics[width=\textwidth]{results/convergence_sin.pdf}
  \caption[Convergence graph \prob{2}]{Convergence graph of different methods on 
  \prob{2} in $L^2$ norm. The 'FEM (ur)'
  data comes from the problem without well solved by classical FEM and with optimal convergence order 2.0.}
  \label{fig:convergence_sin}
\end{figure}

\subsection{Conditioning of the system matrix} \label{sec:res_conditioning}
We shall write here a brief note about the conditioning of the system.
Condition number for matrices resulting from a conforming FEM applied to Laplace equation is $O(h^{-2})$, so the iteration count 
for CG without preconditioning is $O(h^{-1})=O(\sqrt{n})$, where $n=1/h^2$ is number of degrees of freedom in case of linear finite elements. 
With local preconditioning (Jacobi, 
SOR, ILU) one can usually achieve the number of iterations $O(h^{-0.5})$, c.f. \cite{ern_evaluation_2006}.

Let us use the data from the numerical tests above in \ref{sec:res_comparison}.
We observe the iteration count needed by CG solver in \fig{fig:iterations}.
The number of iterations of the standard FEM is corresponding to the classic results as mentioned in the paragraph above. 

\begin{figure}[!htb]
%   \vspace{0pt}
  \centering    
  \includegraphics[width=\textwidth]{results/iterations.pdf}
  \caption[Iterations graph]{Graph of dependence of the CG iteration count on the 
  number of degrees of freedom. Measured on both problems with no serious distinction observed.}
  \label{fig:iterations}
\end{figure}
%
We can see clearly the enormous growth of the number of iterations in case of the standard XFEM and the ramp 
function XFEM. These problems are generally known and are described for example in the overview of the XFEM in
\cite{fries_extended/generalized_2010}. The usage of enrichment functions can make the approximation space almost linearly 
dependent from which the ill-conditioning of the system arises. That is exactly what the SGFEM was developed to deal with.

Summarizing the theory in \cite{babuska_stable_2012}, we can say that the conditioning of the SGFEM system is not worse than that of the 
standard FEM system. We have only the number of iterations on the standard FEM system for comparison, 
but we see from the graph in \fig{fig:iterations} that the results in case of the SGFEM are satisfying.
Number of iterations needed by the shifted XFEM is similarly good. This area of the problem would need deeper 
investigation.


\subsection{Dependence on the enrichment radius}
The aim of this section is twofold: we first validate the estimate \eqref{eq:grad_estimate_on_square}, secondly we study 
the dependence of the error in $L^2$ norm on the enrichment radius numerically and compare it with \eqref{eq:singular_approx_error}.
%
\begin{table}
\begin{center}
\begin{tabular}{crr}
\toprule
% \multicolumn{2}{c}{Item} \\
% \cmidrule(r){1-2}
$h$    & min & max \\
\midrule
$\rfrac{10}{8}$   & 0.97 & 7.1  \\% & 1.38 & 10.0  \\ %& 0.7 & 5.3   \\
$\rfrac{10}{16}$  & 0.99 & 16.4  \\% & 1.40 & 23.1  \\ %& 1.0 & 17.4  \\
$\rfrac{10}{32}$  & 1.00 & 34.4  \\% & 1.41 & 48.7  \\ %& 1.5 & 51.8  \\
$\rfrac{10}{64}$  & 1.00 & 70.3  \\% & 1.41 & 99.5  \\ %& 2.1 & 150   \\
$\rfrac{10}{128}$ & 1.00 (0.756)& 142.0   \\% & 1.41 & 201   \\ %& 3.0 & 427   \\
\bottomrule
\end{tabular}
\caption{Minimal and maximal values of the ratio \eqref{eqn:log_h1_estimate_ratio} for sequence of refined 
meshes with element size $h$. Validation of the estimate \eqref{eq:grad_estimate_on_square}.}
\label{tab:log_h1_estimate}
\end{center}
\end{table}
%
\begin{figure}[!htb]
%   \vspace{0pt}
  \centering    
  \subfloat[$\|\log \vc x - u_h\|^2_{H^1(T)}$ in log scale]{\label{fig:log_estimate_a} 
    \includegraphics[width=0.47\textwidth]{results/log_estimate_h1.pdf} }
  \hspace{0pt}
  \subfloat[the ratio \eqref{eqn:log_h1_estimate_ratio}]{\label{fig:log_estimate_b} 
    \includegraphics[width=0.47\textwidth]{results/log_estimate_ratio.pdf} }
  \caption[Log error estimate.]
  {
  Results of the numerical validation of the estimate \eqref{eq:grad_estimate_on_square}. The elements are left out 
  in the center where the $\log$ singularity is situated and where the function is cut off.
  }
  \label{fig:log_estimate}
\end{figure}
%
Validity of the estimate \eqref{eq:grad_estimate_on_square} is verified by a calculation of the ratio
\begin{equation} \label{eqn:log_h1_estimate_ratio}
\frac{h^{3/2} r^{-2} 12^{-1/2}}{\|u_s - \Pi u_s\|^2_{H^1(T)}},\quad u_s(\vc x) = \log \abs{\vc x}
\end{equation}
on every element $T$ of the sequence of refined meshes using a $5\times5$ Gaussian quadrature for the estimation of the $H^1$ norm. Table 
\ref{tab:log_h1_estimate} reports the minimum and the maximum values of the ratio over all elements of every mesh.
The minimum values are close to 1 independently of $h$ which is in perfect agreement with \eqref{eq:grad_estimate_on_square}.
Moreover, the minimum value is attained on the majority of
elements, see \fig{fig:log_estimate_b}. Both parts of \fig{fig:log_estimate} demonstrate also higher convergence rate on diagonal elements
where the nonlinear term of the bilinear finite elements allows better approximation of the saddle shaped logarithmic surface.

Next, we study the influence of the enrichment radius $R$ on the global $L^2$ error. To this end, we solve \prob{2} with parameter $U=3$
using the shifted XFEM for different mesh steps and different values of $R$.
Let us remind that $O(h^p)$ convergence of the solution in the $H^1$ norm translates to the $O(h^{p+1})$ convergence of the solution in the $L^2$ norm 
for the linear elliptic problems (c.f. \cite[Theorem 19.2]{ciarlet_basic_1991}). According to the estimates \eqref{eq:std_err_estimate}
and \eqref{eq:singular_approx_error}, we expect $O(h^2)$ convergence of $L^2$ norm independently of the enrichment radius. This is 
clearly demonstrated in \fig{fig:radius_conv_1}. For comparison, we plot also the error of the regular part of the solution
\[
  u_r(x,y) = U \sin (\omega x)
\]
solved by standard FEM showing the $O(h^2)$ convergence.
As predicted, the total error diminishes with $R$ but cannot 
drop under the error of $u_r$. By direct computation, we get
\[
  \norm{u_r - \Pi u_r}_{H^1(\Omega)} = \frac{1}{12}UHh\omega^2 + O(h^2),
\]
where $H=100$ is the domain side and $U=3$, $\omega=0.03$. Then according to \eqref{eq:singular_approx_error},
we get the optimal value of the enrichment radius
\[
    R_o \sim \sqrt{\frac{\pi}{6}} h/\norm{u_r - \Pi u_r}_{H^1(\Omega)} \sim 9.2
\]
This value roughly matches a point in the plots of the error as a function of $R$ in
\fig{fig:radius_conv_2}, from which the error is not decreasing anymore.


\begin{figure}[!htb]
%   \vspace{0pt}
  \centering    
  \includegraphics[width=0.9\textwidth]{results/radius_conv_1.pdf}
%   \subfloat[rozdìlený element s vrtem]{\label{fig:adapt_ref_a} 
%     \includegraphics[width=70mm]{\figpath adaptive_ref.pdf} }
%   \hspace{0pt}
%   \subfloat[detail hranice vrtu]{\label{fig:adapt_ref_b} 
%     \includegraphics[width=72mm]{\figpath adaptive_ref_detail.pdf} }
  \caption[Enrichment radius choice.]{Convergence graph for different enrichment radii. The 'FEM (ur)'
  data comes from the problem without well solved by the standard FEM -- it has the optimal convergence order 2.0.}
  \label{fig:radius_conv_1}
\end{figure}
\begin{figure}[!htb]
%   \vspace{0pt}
  \centering    
  \includegraphics[width=0.9\textwidth]{results/radius_conv_2.pdf}
%   \subfloat[rozdìlený element s vrtem]{\label{fig:adapt_ref_a} 
%     \includegraphics[width=70mm]{\figpath adaptive_ref.pdf} }
%   \hspace{0pt}
%   \subfloat[detail hranice vrtu]{\label{fig:adapt_ref_b} 
%     \includegraphics[width=72mm]{\figpath adaptive_ref_detail.pdf} }
  \caption[Enrichment radius choice.]{Dependence of the error on the enrichment radius for different
  element sizes.}
  \label{fig:radius_conv_2}
\end{figure}





\section{Summary}
\label{sec:summary}

We presented a study on the usage of four partition of unity methods -- the XFEM, its corrected version with the ramp 
function, the shifted XFEM, and the SGFEM. The problem setting was inspired by the work \cite{gracie_modelling_2010,craig_using_2011} 
of R.~Gracie and J.R.~Craig but our effort was aimed more in understanding details of the PU methods and their comparison
rather than in computation of complex problems.

We investigated the issue of a sub-optimal convergence order reported in \cite{gracie_modelling_2010}. We revealed the problem
and we suggested a better strategy for the adaptive quadrature. The improvement was confirmed by the numerical
tests in \ref{sec:res_comparison} where we obtained the optimal order of convergence in $L^2$ norm.

All the implemented methods were compared in \ref{sec:res_comparison}. Regarding the XFEM, we saw that at 
least the ramp function must be used in order to optimze the error on the blending elements. 
The shifted XFEM and the SGFEM converged optimally and did not show any difference in the solution precision.

The ill-conditioning of the system matrix was encountered when using the standard XFEM and the ramp function XFEM.
The iteration count was measured.

Finally, the choice of the optimal enrichment radius was studied. The error estimate dependent on the enrichment
radius was derived and it was numerically validated. Furthermore, the optimal enrichment radius was predicted 
for the test problem and it corresponded with the computed data.

\section{Acknowledgement}
This work was supported by the Ministry of Education of the Czech Republic within the SGS project 
no. 21066/115 on the Technical University of Liberec. The paper was supported in part by the Project OP
VaVpI Centre for Nanomaterials, Advanced Technologies  and Innovations
CZ.1.05/2.1.00/01.0005.


%% The Appendices part is started with the command \appendix;
%% appendix sections are then done as normal sections
%% \appendix

%% \section{}
%% \label{}

%% If you have bibdatabase file and want bibtex to generate the
%% bibitems, please use
%%
%%  \bibliographystyle{elsarticle-harv} 
%%  \bibliography{<your bibdatabase>}

%% else use the following coding to input the bibitems directly in the
%% TeX file.

% \begin{thebibliography}{00}
% 
% %% \bibitem[Author(year)]{label}
% %% Text of bibliographic item
% 
% \bibitem[ ()]{}
% 
% \end{thebibliography}
 %\nocite{dip}
 %\bibliographystyle{elsarticle-harv} 
 %\bibliographystyle{elsarticle-num-names} 
 \bibliographystyle{elsarticle-num} 
 \bibliography{citace.bib}
\end{document}


%\endinput
%%
%% End of file `elsarticle-template-harv.tex'.

