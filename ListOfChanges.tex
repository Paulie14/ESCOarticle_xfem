%%
%% Example
%%

\documentclass[a4paper,11pt]{article}

%\usepackage[cp1250]{inputenc}
%\usepackage[czech]{babel}          %Windows
%\usepackage{czech}

\begin{document}


{
\begin{center}
{\Large\bf Revision of manuscript AMC-D-14-03790 
\newline
\newline
\emph{Partition of unity methods for approximation of point water sources in porous media}}
\end{center}
}

We would like to thank to reviewers for their opinions, suggestions and constructive remarks.
We edited our manuscript which resulted in following list of changes:

\begin{enumerate}
\item Revision 1: Add Figure 4 - geometry and solution. %1
\item Revision 2: %2
      We believe that the derivation of the weak form of the complex model is eligible
      although we aimed this article mainly at measuring convergence, PUM comparison and integration technique.
      One reason is that it shows exactly what the differences from the model of Gracie and Craig are.
      Secondly, we think obtaining this weak form is worth the effort, because it displays the elliptic
      character of all the terms which one can make a profit out of while proving existence and uniqueness of the solution.
      Finally this form agrees with our implementation which enables multiple aquifers and wells.
      Therefore, we also added link to GitHub repository with our experimental code in the introduction.
      
      We added some comments in the introduction and the end of \emph{Model} section regarding this matter.
      We also supported arguments for simplification to one aquifer in the beginning of \emph{Discretization}.
\item weighted XFEM %3
\item Revision 4: A paragraph added in section 6.1.%4
\item responce to complex time-dependent model %5

% minor changes
\item Corrected revision 6. %6
\item Corrected revision 7. %7
\item Corrected revision 8. %8
\item Corrected revision 9. %9
\item Corrected revision 10. %10

\item Revision 11: Extended explanation why we can use error estimates for 1D quadrature to estimate error of 2D quadrature in our case. %11

\item Corrected revision 12. %12
\item Corrected revision 13, meaning of $f^{(2n)}$ explained -- derivative. %13
\item Corrected revision 14. %14
\item Corrected revision 15. %15
\item Corrected revision 16. %16

\item Revision 17: We are neglecting the influence of the suggested adaptive integration on the unknown error. 
      At the moment, we are sure that the reason is (as we suspected) that the formulation of the model allows a non-constant
      flow across the well edge, while we are prescribing a constant flow in (29). Equation (29) then cannot be satisfied. %17

\item Revision 18: There was wrong reference to Figure 8, instead of Figure 9, showing dependence of the error on enrichment radius. %18
      This explains also next remark, because Figure 9 was not referenced at all.
      This is now corrected. %18
\item Corrected revision 19 (explained above). %19
\item Corrected resvision 20. %20

%Further corrections...
\item Change the notation of the exterior radius from $R$ to $D$, since we want to reserve $R$ for enrichment radius (section 3.1 and 6.1).
\item Replace 'convergence rate' with 'convergence order' in some places, where we talk strictly about the order.
\item In section 4.1 Instability: correct 'see Figure Figure 3a'.
\item Correct some bibliography items - BibTex vs BibLaTex incompatibilities.

\end{enumerate} 


\end{document}
